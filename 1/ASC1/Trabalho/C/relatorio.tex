\documentclass[titlepage,12pt]{article}   % tipo de documento e tamanho das letras

% os seguintes pacotes estendem a funcionalidade básica:
\usepackage[a4paper, total={16cm, 24cm}]{geometry} % tamanho da pagina e do texto
\usepackage[portuguese]{babel}  % texto lingua portuguesa
\usepackage{tikz}               % diagramas
    \usetikzlibrary{shadows}
\usepackage[colorlinks=true]{hyperref}           % links para partes do documento ou para a web
\usepackage{listings}           % incluir codigo
    \renewcommand\lstlistingname{Código}  % Listing em portugues
    \lstset{numbers=left,
            numberstyle=\tiny,
            numbersep=5pt,
            basicstyle=\footnotesize\ttfamily,
            frame=tb,
            rulesepcolor=\color{gray},
            breaklines=true}
\usepackage{titlepic}           % incluir imagem no titlo

% ----------------------------------------------------------------------------
\title{Trabalho Final - Fase 1}
\author{João Cavaco\\nº 42470 \and António Barroso\\nº 44445}
\date{\today}
\titlepic{\includegraphics[]{logo}}

% ----------------------------------------------------------------------------
\begin{document}

\maketitle
\tableofcontents
% ============================================================================
\section{Introdução}

Nesta primeira fase do trabalho desenvolvemos uma calculadora que opera na notação polaca inversa (RPN - Reverse Polish Notation) em linguagem C. A calculadora lê strings da consola, realiza as operações indicadas, mostra o estado da memória da calculadora na forma de uma pilha e opera sobre números inteiros ou em vírgula flutuante.
% ============================================================================
\section{Pilha}
Para o desenvolvimento da calculadora utilizámos a estrutura de dados Pilha que foi implementada através da sua especificação em:
\begin{lstlisting}[language={C}]
typedef struct stack
{
    int capacity; // capacidade máxima da pilha
    int top;      // indice do array que marca o topo da pilha
    float *array; // array de elementos da pilha
} stack_t;
\end{lstlisting}
e dos vários métodos: 
\begin{description}
  \item[$\bullet$ stack\_new(int capacity)] Aloca espaço na memória para a pilha e inicializa-a
  \item[$\bullet$ stack\_size(stack\_t *stack)] Retorna um int que representa o tamanho da pilha
  \item[$\bullet$ stack\_empty(stack\_t *stack)] Retorna um booleano que indica se a pilha está vazia
  \item[$\bullet$ stack\_full(stack\_t *stack)] Retorna um booleano que indica se a pilha está cheia
  \item[$\bullet$ stack\_push(stack\_t *stack, float x)] Empilha um elemento na pilha 
  \item[$\bullet$ stack\_peek(stack\_t *stack)] Retorna o elemento no topo da pilha 
  \item[$\bullet$ stack\_display(stack\_t *stack)] Mostra o estado da memória da calculadora
  \item[$\bullet$ stack\_pop(stack\_t *stack)] Remove e retorna o elemento no topo da pilha  
  \item[$\bullet$ stack\_destroy(stack\_t *stack)] Remove o espaço alocado para a pilha na memória
\end{description}
% ============================================================================
\section{Funções}

Estas funções estão descritas em maior detalhe nos comentários do programa.

\begin{description}
  \item[$\bullet$ add(stack\_t *stack)] Empilha o resultado da soma dos dois operandos do topo da pilha
  \item[$\bullet$ subtract(stack\_t *stack)] Empilha o resultado da subtracção dos dois operandos do topo da pilha
  \item[$\bullet$ divide(stack\_t *stack)] Empilha o resultado da divisão dos dois operandos do topo da pilha
  \item[$\bullet$ multiply(stack\_t *stack)] Empilha o resultado da multiplicação dos dois operandos do topo da pilha
  \item[$\bullet$ neg(stack\_t *stack)] Transforma o operando no topo da pilha no seu simétrico
  \item[$\bullet$ swap(stack\_t *stack)] Troca a posição dos dois operandos do topo da pilha
  \item[$\bullet$ dup(stack\_t *stack)] Duplica o operando do topo da pilha
  \item[$\bullet$ drop(stack\_t *stack)] Elimina um operando do topo da pilha
  \item[$\bullet$ clear(stack\_t *stack)] Limpa toda a pilha 
  \item[$\bullet$ prompt()] Apresenta um texto no ínicio do programa
  \item[$\bullet$ help()] Apresenta um texto de ajuda 
  \item[$\bullet$ off()] Termina o programa

\end{description}

% ============================================================================
\section{Conclusão}

Esta primeira fase do trabalho foi bastante útil para a nossa aprendizagem pois permitiu-nos aprender um pouco mais sobre a estrutura de dados Pilha e as calculadoras de notação polaca inversa, o que nos ajudará sem dúvida na realização da próxima fase do trabalho.

\end{document}